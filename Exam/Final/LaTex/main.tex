\documentclass{article}

\usepackage{graphicx}
\usepackage{hyperref}
\usepackage{bm}
\usepackage{float}
\restylefloat{table}

\usepackage{listings}
\usepackage{color}
\usepackage{amsmath}

\usepackage[margin=1.25in]{geometry}

\definecolor{dkgreen}{rgb}{0,0.6,0}
\definecolor{gray}{rgb}{0.5,0.5,0.5}
\definecolor{mauve}{rgb}{0.86,0.27,0.22}

\lstset{frame=tb,
  language=python,
  aboveskip=3mm,
  belowskip=3mm,
  showstringspaces=false,
  columns=flexible,
  basicstyle={\small\ttfamily},
  numbers=left,
  stepnumber=1,
  numberstyle=\tiny\color{gray},
  keywordstyle=\color{blue},
  commentstyle=\color{dkgreen},
  stringstyle=\color{mauve},
  breaklines=true,
  breakatwhitespace=true,
  tabsize=3
}

\title{CS5200: Final Exam} % Title of the assignment

\author{Matthew Whitesides\\ \texttt{mbwxd4@umsystem.edu}} % Author name and email address

\date{\today} % University, school and/or department name(s) and a date

%----------------------------------------------------------------------------------------

\begin{document}

  \maketitle % Print the title

  \textit{I, Matthew Whitesides, certify that all the material in this PDF file is my original work, that I did not discuss these questions with anyone other than my instructor, and that I did not copy work from anyone for this examination.}
 
  \begin{enumerate}
    \item \textbf{Hamiltonian Paths and Cycles}.
    
    Please refer to the following labled nodes version of the graph for some of the questions.\\
    \includegraphics[scale=0.75]{1_Graph.png}

    \begin{enumerate}
        \item A Hamiltonian Path is defined as, "a graph path between two vertices of a graph that visits each vertex exactly once". Which yes this graph does have at least one. 
        
        For example the path in the labeled version above: 
        
        ${A \rightarrow F \rightarrow H \rightarrow I \rightarrow B \rightarrow C \rightarrow G \rightarrow J \rightarrow K \rightarrow D \rightarrow L}$ 
        
        would be a Hamiltonian Path.

        \item A Hamiltonian Cycle is defined as, " is a graph cycle (i.e., closed loop) through a graph that visits each node exactly once.". Which yes this graph does have at least one. 
        
        
    \end{enumerate}

    \item \textbf{General Graph Theory}.
    \begin{enumerate}
        \item ToDo.
        \item Let's try a bit of a backword induction approach.
        
        If there are $N = 2$ people at the party then each person would have shaken hands with $(N - 1)$ person, e.g. the person other than John shook hands with one person which must be John.

        So if $Q(n)$ is out problem given $n$ people how many hands did John shake then $Q(2) = 1$.

        Now if $N = 3$ than we have John and two others which at most could have shaken hands with $(N - 1)$ person so the other handshakes would be $(N - 1), (N - 2)$ or $(2),(1)$.

        
        
    \end{enumerate}

    \item \textbf{NP Complete Problems}.
    \item \textbf{Red-Black Trees, B-Trees, and Binary Search Trees}.
    \item \textbf{More Graph Algorithms}.
    \item \textbf{GCD}.
  \end{enumerate}  % End of questions.
\end{document}
