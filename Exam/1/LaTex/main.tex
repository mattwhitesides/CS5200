\documentclass{article}

\usepackage{graphicx}
\usepackage{hyperref}
\usepackage{bm}

\usepackage{listings}
\usepackage{color}

\usepackage[margin=1.25in]{geometry}

\definecolor{dkgreen}{rgb}{0,0.6,0}
\definecolor{gray}{rgb}{0.5,0.5,0.5}
\definecolor{mauve}{rgb}{0.86,0.27,0.22}

\lstset{frame=tb,
  language=python,
  aboveskip=3mm,
  belowskip=3mm,
  showstringspaces=false,
  columns=flexible,
  basicstyle={\small\ttfamily},
  numbers=none,
  numberstyle=\tiny\color{gray},
  keywordstyle=\color{blue},
  commentstyle=\color{dkgreen},
  stringstyle=\color{mauve},
  breaklines=true,
  breakatwhitespace=true,
  tabsize=3
}

%----------------------------------------------------------------------------------------
%	ASSIGNMENT INFORMATION
%----------------------------------------------------------------------------------------

\title{CS5200: Exam \#1} % Title of the assignment

\author{Matthew Whitesides\\ \texttt{mbwxd4@mst.edu}} % Author name and email address

\date{\today} % University, school and/or department name(s) and a date

%----------------------------------------------------------------------------------------

\begin{document}

  \maketitle % Print the title
  
  \textit{I, Matthew Whitesides (Prelim1 ID: 6352198926), certify that all the material in this PDF file is my original work, that I did not discuss these questions with anyone other than my instructor, and that I did not copy work from anyone for this examination.}

  \begin{enumerate}
    \item \textbf{Big O Notation.}
    \begin{enumerate}
      \item We want to show that $81n^3 + 1300n^2 + 300n \in O(n^5 - 15000n^4 - 10n^3)$. We have a much larger exponent with 5 but obviously the second part won't dominate for a while with the -15000 as it isn't even positive until n $\ge 15000$.
      \begin{itemize}
        \item I'll pick $c = 1$ and $n_0 = 1000$.
        \item Note if we have $n \geq 1000$ we have:
        \item $81n^3 + 1300n^2 + 300n \le 100n^3 + 10000n^2 + 1000n$.
        \item $100n^3 + 10000n^2 + 1000n \le n^4 \;\forall\; n \geq 1000$.
          \begin{itemize}
            \item Because $n^4$ is at least $1000(n^3)$ for $n \geq 1000$.
            \item $100(n^3)$ is at least $10000n^2$ for $n \geq 100$. 
            \item And $100(n^2)$ is at least $10000n$ for $n \geq 100$.
          \end{itemize}
          \item Now we just have to prove $n^4 \leq n^5 - 15000n^4 - 10n^3 \;\forall\; n \geq 100000$.
          \item Note: I have now picked $n \geq 100000$ which will be part of the final proof.
          \begin{itemize}
            \item If we subtract $n^4$ from both sides we get \\ $0 \leq n^5 - 15001n^4 - 10n^3 \;\forall\; n \geq 100000$.
            \item And if we add $15001n^4$ and $10n^3$ to both sides we get \\ $15001n^4 + 10n^3 \leq n^5 \;\forall\; n \geq 100000$.
            \item Because $100000n^4 \leq n^5 \;\forall\; n \geq 100000$. \\
                  And $10n^3 \leq (10000)n^4 \;\forall\; n \geq 0$.
          \end{itemize}
          \item Therefore we have proven that: \\
          $81n^3 + 1300n^2 + 300n \in O(n^5 - 15000n^4 - 10n^3)\;\forall\;n \geq 100000$ \\
          As the first part dominates the second for all values greater then our threshold.
          \item Also due to the definition of Big O the inverse:\\ ($n^5 - 15000n^4 - 10n^3 \notin O(81n^3 + 1300n^2 + 300n)$) \\ must be true because at no point can the first part dominate the other. 
      \end{itemize}
      \item In this particular case I don't believe you can say $f(x) \in g(x)$ or $g(x) \in f(x)$ due to the nature of the sin and cos relationship.
      \begin{itemize}
        \item For example take $f(\pi) = 0$ and $g(\pi) = -3$ for these values you could say $f(\pi)$ dominates $g(\pi)$.
        \item However now take $f(2\pi) = 0$ and $g(2\pi) = 7$ for these values you could say $g(2\pi)$ dominates $f(2\pi)$.
        \item Because of them begin sin and cos waves this pattern will continue on indefinately $f(3\pi) = 0$ and $g(3\pi) = -3$, $f(4\pi) = 0$ and $g(4\pi) = 7$, etc.
        \item Meaning that due to the definition of Big O for no indefinately increasing values of x will $f(x)$ dominate $g(x)$ or $g(x)$ dominate $f(x)$.
      \end{itemize}
    \end{enumerate}

    \item \textbf{Simple Graph Algorithms.} See GraphDataOut.txt and whitesidesMatthew\_Exam1\_Q2.py.
    
    \item \textbf{Proof by Induction.}
    \begin{enumerate}
      \item Relationship.
      \begin{enumerate}
        \item So essentially $\textrm{sf}(n) = \textrm{f}(n) + \textrm{f}(n - 1) + \textrm{f}(n - 2) + ...\;\textrm{f}(n - n)$.
        \item Which for f(0) and f(1) are the two base cases for f, which return 1000 and 7000 respectively and add them together. So we're looking for to fib sequences that add up to sf.
        \item Let's define Fibonacci as Binet's forumla $fib(n) = \frac{(1+\sqrt{5})^n-(1-\sqrt{5})^n}{2^n \sqrt{5}}$.
        \item Thus our final formula for sf's relation to f is:\\ $\textrm{sf}(n) = (fib(n + 1) * f(0)) + (((fib(n + 2) - 1) * f(1)))$.
      \end{enumerate}
      
      \item Proof.        
      \begin{enumerate}
        \item Define the Problem.
          \begin{itemize}
            \item The domain for this example is all positive integers greater than 0. 
            \item A value of 0 will always return 1000 and less than will always return 0, and won't make senese for our proof.
            \item Thus 1 is the base case.
            \item Essentailly function sf translates into recursive calls to $\textrm{sf}(n) = \textrm{f}(n - 1) + \textrm{f}(n - 2) + \textrm{f}(n - 3) + ...\;\textrm{f}(n - n)$?
            \item For each integer i between 0 and n, $\textrm{sf}(n)$ is equilivent to the sum of each individual call to $\textrm{f}(i)$.
            \item Now essentailly we see that we have a Fibonacci sequence for f(n) is the result of sf.
            \item Due to the nature of f we will return 1000 for f(0) and 7000 for f(1), which also will recursivly call itself to return the product of f(n - 1) and f(n -2) hence it inand of itself is another layer of recursion. 
            \item To translate that into a linear equation we need to add together values of when f(0) and f(1) is triggered which are both triggered fib(n + x) times of some value of x.
            \item For f(0) we hit it fib(n + 1) times because for the first two values of fib we don't recur we just return a constant therefor we need to start our sequece at 1 above the value of n.
            \item For f(1) we hit it fib(n + 2) - 1 times because due to f()'s recurrsion to call n - 2 there is going to one more case called then f(0). Also we subtract 1 from the fib sequence because on the last case of n = 1 we call n of 0 and -1 not fib(1).
            \item Also we multiple each of side of our formula by the values that f returns, 1000 and 7000 respectively.
            \item Giving us $\textrm{sf}(n) = P(n) = (fib(n + 1) * f(0)) + (((fib(n + 2) - 1) * f(1)))$.
          \end{itemize}
        
        \item Check the Stopping Values and Two Others.
          \begin{itemize}
            \item 1 is the only stopping condition for the function $\textrm{sf}(n)$ given our domain of n only checking positive integers greater than 0.
            \item $\textrm{sf}(n)$ will stop only if n is -1 which will occur when $\textrm{sf}(0)$ is called, it will recur to call $\textrm{sf}(-1)$ but that's outside the relationshop with f().
            \item The same with f(0) which is outside our proof.
            \item What is $\textrm{sf}(1)$? It is 8000 why?
            \begin{itemize}
              \item Because $\textrm{P}(1) = (fib(1 + 1) * f(0)) + ((fib(1 + 2) - 1) * f(1))$ \\
              $ = (1 * f(0)) + (((2 - 1) * f(1)))$ \\
              $ = (1 * 1000) + (1 * 7000) = 8000$
            \end{itemize}
            \item What is $\textrm{sf}(2)$? It is 16000 why?
            \begin{itemize}
              \item Because $\textrm{P}(2) = (fib(2 + 1) * f(0)) + ((fib(2 + 2) - 1) * f(1))$ \\
              $ = (2 * f(0)) + ((3 - 1) * f(1))$ \\
              $ = (2 * 1000) + (2 * 7000) = 16000$
            \end{itemize}
            \item What is $\textrm{sf}(8)$? It is 412000 why?
            \begin{itemize}
              \item Because $\textrm{P}(8) = (fib(8 + 1) * f(0)) + ((fib(8 + 2) - 1) * f(1))$ \\
              $ = (34 * f(0)) + ((55 - 1) * f(1))$ \\
              $ = (34 * 1000) + (54 * 7000) = 412000$
            \end{itemize}
            \item What is $\textrm{sf}(9)$? It is 671000 why?
            \begin{itemize}
              \item Because $\textrm{P}(9) = (fib(9 + 1) * f(0)) + ((fib(9 + 2) - 1) * f(1))$ \\
              $ = (55 * f(0)) + ((89 - 1) * f(1))$ \\
              $ = (55 * 1000) + (88 * 7000) = 671000$
            \end{itemize}
          \end{itemize}

        \item Prove that for all $n > s$, that if P(n - 1) is true then P(n) is true.
          \begin{itemize}
            \item We assume P(n - 1) is true.
            \item $\textrm{sf}(n - 1) = g(n - 1)$.
            \item $\textrm{sf}(n) = \textrm{sf}(n - 1) + \textrm{sf}(n - 2)$ Definition of sf.
            \item $ = \textrm{g}(n - 1) + \textrm{sf}(n - 2)$ If P(n - 1) is true.
            \item $ = (fib((n - 1) + 1) * f(0)) + (((fib((n - 2) + 2) - 1) * f(1))) + \textrm{sf}(n - 2) + ((fib((n - 2) + 1) * f(0)) + (((fib((n - 2) + 2) - 1) * f(1))))$ Definition of g().
            \item $ = (fib(n + 1) * f(0)) + (((fib(n + 2) - 1) * f(1)))$
          \end{itemize}

        \item Conclusion.
          \begin{itemize}
            \item We have shown that for all positive integers that $\textrm{sf}(n) = P(n) = (fib(n + 1) * f(0)) + (((fib(n + 2) - 1) * f(1)))$ is true.
          \end{itemize}          
      \end{enumerate}
    \end{enumerate}

    \item Probability.
    \begin{enumerate}
      \item Sample space of:\\ 
      $Vec_3 =$ \{\\
        (0, 0, 0), (0, 0, 1), (0, 0, 2), \\
        (0, 1, 0), (0, 1, 1), (0, 1, 2), \\
        (0, 2, 0), (0, 2, 1), (0, 2, 2), \\
        (1, 0, 0), (1, 0, 1), (1, 0, 2), \\
        (1, 1, 0), (1, 1, 1), (1, 1, 2), \\
        (1, 2, 0), (1, 2, 1), (1, 2, 2), \\
        (2, 0, 0), (2, 0, 1), (2, 0, 2), \\
        (2, 1, 0), (2, 1, 1), (2, 1, 2), \\
        (2, 2, 0), (2, 2, 1), (2, 2, 2)\} 

        \pagebreak

        Here are the possible quirks in $Vec_3$ with at total comming in a 27: \\
        (0, 1, 0), = [1,0], (0, 2, 0), = [2,0], (0, 2, 1), = [2,1], (1, 0, 0), = [1,0],
        (1, 0, 0), = [1,0], (1, 0, 1), = [1,0], (1, 0, 2), = [1,0], (1, 1, 0), = [1,0],
        (1, 1, 0), = [1,0], (1, 2, 0), = [1,0], (1, 2, 0), = [2,0], (1, 2, 1), = [2,1],
        (2, 0, 0), = [2,0], (2, 0, 0), = [2,0], (2, 0, 1), = [2,0], (2, 0, 1), = [2,1],
        (2, 0, 2), = [2,0], (2, 1, 0), = [2,1], (2, 1, 0), = [2,0], (2, 1, 0), = [1,0], 
        (2, 1, 1), = [2,1], (2, 1, 1), = [2,1], (2, 1, 2), = [2,1], (2, 2, 0), = [2,0],
        (2, 2, 0), = [2,0], (2, 2, 1), = [2,1], (2, 2, 1), = [2,1]

        \item The uniform probabilty for any space $v \in Vec_n = \frac{1}{n^n}$.
        Now we can elaborate this out to understand that given we have $n^n$ options of spaces and each v has n options within it, using the calculated quirks in $Vec_3$ of 27 we can say:\\
        $Quirks \in Vec_n = n^n$.        
      \end{enumerate}

      \item First lets look at the probabilty of getting no heads when you flip a coin n times. 
      Given a simple sample space of 2 flips we have $|S| = {TT, TH, HT, HH}$ where only one options gives us the no heads. 
      Therefore we can extract that the no heads happens $\frac{1}{2^n}$ times. \\
      Now lets look at flipping a coin fewer then n heads when flipping 4n times. 
      The simple case of one would be what's the probablity of getting fewer then 1 heads when you flip a coin (4*1) times.
      Well the only case is obviously also the same as 0 heads which is $\frac{1}{2^4}$ times. 
      Given that literally there's only ever one case possible to get no heads and even the base case of n = 1 for 4n flips counts twords the probabilty of the second case they at the least are of equal probabilty.
      However we know that other than just the single case we have multiple options for flipping less than n heads 4n times.

      Look at just the first couple probabilty of how to get differnt ammounts of heads flipping a coin 4*2 times. \\

      \begin{tabular}{lll}
        Num Heads & Ways to no Heads & Probability \\
        0         & 1                 & 1/256       \\
        \end{tabular} \\

      \begin{tabular}{lll}
      Num Heads & Ways to less than 2 Heads & Probability \\
      0         & 1                 & 1/256       \\
      1         & 28                & 28/256     
      \end{tabular} \\

      And we can actually stop there because if we add up the probabiltyof getting 0 or 1 heads we get 29/256 which is greater then the probabilty of no heads (1/256).
      As the number n grows the number of ways to make fewer than n heads grows but the number of ways to get no heads is always $\frac{1}{2^n}$ which always at least that. Which the total probabilty $sum(n - 1 heads) + sum(n - 2 heads) ... sum(n - n heads or 0 heads)$.
      Thus aside from n = 1 which is equal to the probabilty of no heads, P of fewer than n heads is always greater than P of no heads.

      \item Combinatorics
      \begin{enumerate}
        \item test
      \end{enumerate}

      \item To help understand s(n) it will help to look at the equation for f(n). \\
      $f(n) = 3n(3n - 1)(3n - 2)$ which expands to $27n^3 - 27n^2 + 6n$ (using algebra). \\
      And all that s(n) is doing is recursivly calling $s(n) = f(n) + f(n - 1) + ... f(n - n)$.\\
      So essentailly that means s(n) = $\sum_{i=0}^n (27(n - i)^3 - 27(n - i)^2 + 6(n - i))$ \\
      Now since we have essentailly a fixed formula we can treat them like the natural numbers idea and multiply each coefficient by $n(n+1)/2$.\\
      Giving us \bm{$(27 n^4)/4 + (9 n^3)/2 - (15 n^2)/4 - (3 n)/2$}.\\

      Let's test the equation on range(5):\\
      \begin{tabular}{lll}
        \textbf{n} & \textbf{Our Equation} & \textbf{s(n)} \\
        0 & $(27 (0)^4)/4 + (9 (0)^3)/2 - (15 (0)^2)/4 - (3 (0))/2 = 0$ & 0       \\
        1 & $(27 (1)^4)/4 + (9 (1)^3)/2 - (15 (1)^2)/4 - (3 (1))/2 = 6$ & 6       \\
        2 & $(27 (2)^4)/4 + (9 (2)^3)/2 - (15 (2)^2)/4 - (3 (2))/2 = 126$ & 126       \\
        3 & $(27 (3)^4)/4 + (9 (3)^3)/2 - (15 (3)^2)/4 - (3 (3))/2 = 630$ & 630       \\
        4 & $(27 (4)^4)/4 + (9 (4)^3)/2 - (15 (4)^2)/4 - (3 (4))/2 = 1950$ & 1950       \\
        \end{tabular} \\

        \item \textbf{Logical Reasoning} 
        \begin{enumerate}
          \item Lets say that we have G "good" chips and B "bad" chips which are more than half of the chips. Both adds up to n number of chips. 
          And for simplicity sake we're saying the B chips will always give the wrong answer. 
          For example if you had two bad chips both chips will always say the other is good, or if you had just one bad chip it would say the other is bad and the good one would say the other is bad.
          Thererfore if you have more then 50\% bad chips in case one you cannot tell if you had two good chips or two lieing bad chips and in every other case you would have no way of knowing which chip is telling the truth about the other being bad.

          \item So we're going to pair up two chips of floor(n/2) meaning that if we have an odd number it dosn't get paired. 
          With two pairs of chips which we don't know which one is wich we have essentially three choices (remember G = good, B = bad).\\
          
          \begin{tabular}{lll}
            \textbf{Case} & \textbf{Outcome} & \textbf{Action} \\
            1 & GG & Randomly pick one. \\
            2 & GB & Pick the "good" one. \\
            3 & BB & Discard both. \\
            \end{tabular} \\

          Since we are saying that there's at least (n/2) + 1 G chips if we consider what would happen in each case.\\
          1: Worst case two actually bad chips say they're two good chips, in that case we still throw away at least one bad chip otherwise we get one good chip (1/2) odds. \\
          2: Worst case the bad chip would have to say the good chip is bad in which case we know we have a good chip, otherwise we may have two bad chips however picking the G one will still give better than (n/2) odds. (This ones the key to knowing we'll get at least one good one) \\
          3: In this we know we have at least one bad chip and possibly two bad ones so we know we'll throw away a bad and also potentially thow away a good. \\

          Thus if we use the recursive pairwise test we will narrow our pair's of chips down to at least one good one, nowing in the n set the num good is greater than num bad. \\

          If using our test we end up with two chips left we just pairwise the test again, if we have an odd number of chips. 

          \item If we use our chip we found in question to to know we can test all the other chips we have $\textrm{O}(n)$ + finding the good chip. 
          And our time complexity of finding the good chip is (n/2) so essentailly we have $T(n) = T(n/2) + n$ which by using the master method we get a = 1, b = 2, f(n) = n. Which actually we can easily see that n is larger than n/2 so we can go with case 3 and say our recurrence is \bm{$\theta(n)$}.

        \end{enumerate}
  \end{enumerate}
\end{document}
