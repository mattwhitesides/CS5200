\documentclass{article}

\usepackage{graphicx}
\usepackage{hyperref}
\usepackage{bm}

\usepackage{listings}
\usepackage{color}

\definecolor{dkgreen}{rgb}{0,0.6,0}
\definecolor{gray}{rgb}{0.5,0.5,0.5}
\definecolor{mauve}{rgb}{0.86,0.27,0.22}

\lstset{frame=tb,
  language=python,
  aboveskip=3mm,
  belowskip=3mm,
  showstringspaces=false,
  columns=flexible,
  basicstyle={\small\ttfamily},
  numbers=none,
  numberstyle=\tiny\color{gray},
  keywordstyle=\color{blue},
  commentstyle=\color{dkgreen},
  stringstyle=\color{mauve},
  breaklines=true,
  breakatwhitespace=true,
  tabsize=3
}

%----------------------------------------------------------------------------------------
%	ASSIGNMENT INFORMATION
%----------------------------------------------------------------------------------------

\title{CS5200: Exam \#1} % Title of the assignment

\author{Matthew Whitesides\\ \texttt{mbwxd4@mst.edu}} % Author name and email address

\date{\today} % University, school and/or department name(s) and a date

%----------------------------------------------------------------------------------------

\begin{document}

  \maketitle % Print the title
  
  \textit{I, Matthew Whitesides (Prelim1 ID: 6352198926), certify that all the material in this PDF file is my original work, that I did not discuss these questions with anyone other than my instructor, and that I did not copy work from anyone for this examination.}

  \begin{enumerate}
    \item \textbf{Big O Notation.}
    \begin{enumerate}
      \item We want to show that $81n^3 + 1300n^2 + 300n \in O(n^5 - 15000n^4 - 10n^3)$. \\
            We have a much larger exponent with 5 but obviously the second part won't dominate for a while with the -15000 as it isn't even positive until n $\ge 15000$.
      \begin{itemize}
        \item I'll pick $c = 1$ and $n_0 = 1000$.
        \item Note if we have $n \geq 1000$ we have:
        \item $81n^3 + 1300n^2 + 300n \le 100n^3 + 10000n^2 + 1000n$.
        \item $100n^3 + 10000n^2 + 1000n \le n^4 \;\forall\; n \geq 1000$.
          \begin{itemize}
            \item Because $n^4$ is at least $1000(n^3)$ for $n \geq 1000$.
            \item $100(n^3)$ is at least $10000n^2$ for $n \geq 100$. 
            \item And $100(n^2)$ is at least $10000n$ for $n \geq 100$.
          \end{itemize}
          \item Now we just have to prove $n^4 \leq n^5 - 15000n^4 - 10n^3 \;\forall\; n \geq 100000$.
          \item Note: I have now picked $n \geq 100000$ which will be part of the final proof.
          \begin{itemize}
            \item If we subtract $n^4$ from both sides we get \\ $0 \leq n^5 - 15001n^4 - 10n^3 \;\forall\; n \geq 100000$.
            \item And if we add $15001n^4$ and $10n^3$ to both sides we get \\ $15001n^4 + 10n^3 \leq n^5 \;\forall\; n \geq 100000$
            \item Because $100000n^4 \leq n^5 \;\forall\; n \geq 100000$ \\
                  And $10n^3 \leq (10000)n^4 \;\forall\; n \geq 0$
          \end{itemize}
          \item Therefore we have proven that: \\
          $81n^3 + 1300n^2 + 300n \in O(n^5 - 15000n^4 - 10n^3)\;\forall\;n \geq 100000$ \\
          As the first part dominates the second for all values greater then our threshold.
          \item Also due to the definition of Big O the inverse:\\ ($n^5 - 15000n^4 - 10n^3 \notin O(81n^3 + 1300n^2 + 300n)$) \\ must be true as at no point can one dominate the other due. 
      \end{itemize}
      \item 
    \end{enumerate}
  \end{enumerate}  
  
\end{document}
