\documentclass{article}

\usepackage{graphicx}
\usepackage{hyperref}
\usepackage{bm}

\usepackage{listings}
\usepackage{color}

\usepackage[margin=1.25in]{geometry}

\definecolor{dkgreen}{rgb}{0,0.6,0}
\definecolor{gray}{rgb}{0.5,0.5,0.5}
\definecolor{mauve}{rgb}{0.86,0.27,0.22}

\lstset{frame=tb,
  language=python,
  aboveskip=3mm,
  belowskip=3mm,
  showstringspaces=false,
  columns=flexible,
  basicstyle={\small\ttfamily},
  numbers=none,
  numberstyle=\tiny\color{gray},
  keywordstyle=\color{blue},
  commentstyle=\color{dkgreen},
  stringstyle=\color{mauve},
  breaklines=true,
  breakatwhitespace=true,
  tabsize=3
}

%----------------------------------------------------------------------------------------
%	ASSIGNMENT INFORMATION
%----------------------------------------------------------------------------------------

\title{CS5200: Exam \#1} % Title of the assignment

\author{Matthew Whitesides\\ \texttt{mbwxd4@mst.edu}} % Author name and email address

\date{\today} % University, school and/or department name(s) and a date

%----------------------------------------------------------------------------------------

\begin{document}

  \maketitle % Print the title
  
  \textit{I, Matthew Whitesides (Prelim1 ID: 6352198926), certify that all the material in this PDF file is my original work, that I did not discuss these questions with anyone other than my instructor, and that I did not copy work from anyone for this examination.}

  \begin{enumerate}
    \item \textbf{Big O Notation.}
    \begin{enumerate}
      \item We want to show that $81n^3 + 1300n^2 + 300n \in O(n^5 - 15000n^4 - 10n^3)$. We have a much larger exponent with 5 but obviously the second part won't dominate for a while with the -15000 as it isn't even positive until n $\ge 15000$.
      \begin{itemize}
        \item I'll pick $c = 1$ and $n_0 = 1000$.
        \item Note if we have $n \geq 1000$ we have:
        \item $81n^3 + 1300n^2 + 300n \le 100n^3 + 10000n^2 + 1000n$.
        \item $100n^3 + 10000n^2 + 1000n \le n^4 \;\forall\; n \geq 1000$.
          \begin{itemize}
            \item Because $n^4$ is at least $1000(n^3)$ for $n \geq 1000$.
            \item $100(n^3)$ is at least $10000n^2$ for $n \geq 100$. 
            \item And $100(n^2)$ is at least $10000n$ for $n \geq 100$.
          \end{itemize}
          \item Now we just have to prove $n^4 \leq n^5 - 15000n^4 - 10n^3 \;\forall\; n \geq 100000$.
          \item Note: I have now picked $n \geq 100000$ which will be part of the final proof.
          \begin{itemize}
            \item If we subtract $n^4$ from both sides we get \\ $0 \leq n^5 - 15001n^4 - 10n^3 \;\forall\; n \geq 100000$.
            \item And if we add $15001n^4$ and $10n^3$ to both sides we get \\ $15001n^4 + 10n^3 \leq n^5 \;\forall\; n \geq 100000$.
            \item Because $100000n^4 \leq n^5 \;\forall\; n \geq 100000$. \\
                  And $10n^3 \leq (10000)n^4 \;\forall\; n \geq 0$.
          \end{itemize}
          \item Therefore we have proven that: \\
          $81n^3 + 1300n^2 + 300n \in O(n^5 - 15000n^4 - 10n^3)\;\forall\;n \geq 100000$ \\
          As the first part dominates the second for all values greater then our threshold.
          \item Also due to the definition of Big O the inverse:\\ ($n^5 - 15000n^4 - 10n^3 \notin O(81n^3 + 1300n^2 + 300n)$) \\ must be true because at no point can the first part dominate the other. 
      \end{itemize}
      \item In this particular case I don't believe you can say $f(x) \in g(x)$ or $g(x) \in f(x)$ due to the nature of the sin and cos relationship.
      \begin{itemize}
        \item For example take $f(\pi) = 0$ and $g(\pi) = -3$ for these values you could say $f(\pi)$ dominates $g(\pi)$.
        \item However now take $f(2\pi) = 0$ and $g(2\pi) = 7$ for these values you could say $g(2\pi)$ dominates $f(2\pi)$.
        \item Because of them begin sin and cos waves this pattern will continue on indefinately $f(3\pi) = 0$ and $g(3\pi) = -3$, $f(4\pi) = 0$ and $g(4\pi) = 7$, etc.
        \item Meaning that due to the definition of Big O for no indefinately increasing values of x will $f(x)$ dominate $g(x)$ or $g(x)$ dominate $f(x)$.
      \end{itemize}
    \end{enumerate}

    \item \textbf{Simple Graph Algorithms.} See GraphDataOut.txt and whitesidesMatthew\_Exam1\_Q2.py.
    \item \textbf{Proof by Induction.}
    \begin{enumerate}
      \item So essentially $\textrm{sf}(n) = \textrm{f}(n) + \textrm{f}(n - 1) + \textrm{f}(n - 2) + ...\;\textrm{f}(n - n)$.
      \item Proof.
      \begin{enumerate}
        \item Define the problem.
          \begin{itemize}
            \item The domain for this example is all positive integers. 
            \item Any value of $n < 0$ always returns 0.
            \item Thus 0 is the base case.
            \item What does it mean for $\textrm{sf}(n) = \textrm{f}(n - 1) + \textrm{f}(n - 2) + \textrm{f}(n - 3) + ...\;\textrm{f}(n - n)$?
            \item For each integer i between 0 and n, $\textrm{sf}(n)$ is equilivent to the sum of each individual call to $\textrm{f}(i)$.
            \item For Example:
            \begin{itemize}
              \item If $n = 0$: $\textrm{sf}(0) = \textrm{f}(0) = 1000$
              \item If $n = 1$: $\textrm{sf}(1) = \textrm{f}(1) + \textrm{f}(0) = 8000$
              \item If $n = 2$: $\textrm{sf}(2) = \textrm{f}(2) + \textrm{f}(1) + \textrm{f}(0) = 16000$
              \item If $n = 3$: $\textrm{sf}(3) = \textrm{f}(3) + \textrm{f}(2) + \textrm{f}(1) + \textrm{f}(0) = 31000$
            \end{itemize}
            \item Proving the values of $\textrm{f}(n)$ is beyond the scope of this proof. 
          \end{itemize}
        
        \item Check the Stopping Values and Two Others.
          \begin{itemize}
            \item There is only one stopping condition for the function $\textrm{sf}(n)$ given our domain of n only checking positive integers.
            \item $\textrm{sf}(n)$ will stop only if n is -1 which will occur when $\textrm{sf}(0)$ is called, it will recur to call $\textrm{sf}(-1)$.
            \item What is $\textrm{sf}(0)$? It is 1000 why?
            \begin{itemize}
              \item Because $\textrm{f}(0) = 1000$ due to the nature of f.          
            \end{itemize}
            \item What is $\textrm{sf}(1)$? It is 8000 why?
            \begin{itemize}
              \item Because $\textrm{f}(0) = 1000$ and $\textrm{f}(1) = 7000$ due to the nature of f. Thus sf is the sum of 1000 and 7000.
            \end{itemize}
            \item What is $\textrm{sf}(2)$? It is 16000 why?
            \begin{itemize}
              \item Because $\textrm{f}(0) = 1000$ and $\textrm{f}(1) = 7000$ and $\textrm{f}(2) = 8000$ due to the nature of f. Thus sf is the sum of 1000, 7000 and 8000.
            \end{itemize}
          \end{itemize}
        
        \item Check that Recursion stays in D.
          \begin{itemize}
            \item What is D? It is all integers from 0 to $\infty$. 
            \item If recursion is called $\textrm{sf}(n)$ calls $\textrm{sf}(n - 1)$.
            \item Why is $n - 1$ in D? Because if $n - 1 < 0$ recursion is not called.
            \item Thus n is always in domain D.
          \end{itemize}

        \item Show that Recursion Halts.
          \begin{itemize}
            \item We can easily use n as a counter.
            \item Does n always decrease when a recursive call is made? Yes because sf always calls (n - 1).
            \item This means recursion halts.
          \end{itemize}
        \end{itemize}

        \item Show that f is Inherited.
          \begin{itemize}
            \item We assume that f(n - 1) is true.
            \item 
            \item 
          \end{itemize}          
      \end{enumerate}
    \end{enumerate}
  \item 
  \end{enumerate}    
\end{document}
