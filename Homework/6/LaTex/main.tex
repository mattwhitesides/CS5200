\documentclass{article}

\usepackage{graphicx}
\usepackage{hyperref}
\usepackage{bm}
\usepackage{float}
\restylefloat{table}

\usepackage{listings}
\usepackage{color}
\usepackage{amsmath}

\usepackage[margin=1.25in]{geometry}

\definecolor{dkgreen}{rgb}{0,0.6,0}
\definecolor{gray}{rgb}{0.5,0.5,0.5}
\definecolor{mauve}{rgb}{0.86,0.27,0.22}

\lstset{frame=tb,
  language=python,
  aboveskip=3mm,
  belowskip=3mm,
  showstringspaces=false,
  columns=flexible,
  basicstyle={\small\ttfamily},
  numbers=left,
  stepnumber=1,
  numberstyle=\tiny\color{gray},
  keywordstyle=\color{blue},
  commentstyle=\color{dkgreen},
  stringstyle=\color{mauve},
  breaklines=true,
  breakatwhitespace=true,
  tabsize=3
}

%----------------------------------------------------------------------------------------
%	ASSIGNMENT INFORMATION
%----------------------------------------------------------------------------------------

\title{CS5200: Homework \#6} % Title of the assignment

\author{Matthew Whitesides\\ \texttt{mbwxd4@umsystem.edu}} % Author name and email address

\date{\today} % University, school and/or department name(s) and a date

%----------------------------------------------------------------------------------------

\begin{document}

  \maketitle % Print the title
 
  \begin{enumerate}
    \item \textbf{22.1-1 p. 592}.
    
    Essentailly to calculate an in-degree list of the vertecies in an adjacency-list graph we'd just need to iterate over each vertex $O(V)$.
    Then iterate through the linked list attatched to each vertex $O(E)$ and countt the number of vertecies.
    Therefore the total time complexity would be \bm{$O(V + E)$} where V is number of vertecies and E is number of edges comming out of the vertex.

    To calculate an out-degree list of the vertecies we'd also need to iterate over each vertex $O(V)$.
    Then iterate through the linked list attatched to each vertex $O(E)$ however this time we are counting each vertex in the edge list and adding them to our count in the vertex in list.
    Therefore the total time complexity would still be \bm{$O(V + E)$} where V is number of vertecies and E is number of edges going into each vertex in the edge list.

    A simple python program to simulate this is like so:

    \begin{lstlisting}
# A function for given a graph dictionary g we calculate the outdegree of each vertex.
def get_out_degree(g):
    out_degrees = {}
    # O(V) Time
    for v in g:
        out_degrees[v] = 0
        # O(Ei) Time note this is iterating the list even thoguh we have them in memory.
        for ei in g[v]:
            out_degrees[v] += 1 

    # Thus final time complexity is O(V + E)
    return out_degrees


def get_in_degree(g):
    in_degrees = {}
    # O(V) Time
    for v in g:
        in_degrees[v] = 0
        
    for v in g:
        # O(Ei) Time note this is iterating the list even thoguh we have them in memory.
        # This is essentailly the same as the previous except it's adding from the in list.
        for ei in g[v]:
            in_degrees[ei] += 1 

    # Thus final time complexity is still O(V + E)
    return in_degrees              
    \end{lstlisting}

    \begin{lstlisting}
adj_list = { "a": ["c", "f"],
  "b": ["c", "e"],
  "c": ["a", "b", "d", "e"],
  "d": ["c", "a", "f"],
  "e": ["c", "b"],
  "f": []
}

print("Out degrees:", get_out_degree(adj_list))
print("In degrees:", get_in_degree(adj_list))

# Output:
# Out degrees: {'a': 2, 'b': 2, 'c': 4, 'd': 3, 'e': 2, 'f': 0}
# In degrees: {'a': 2, 'b': 2, 'c': 4, 'd': 1, 'e': 2, 'f': 2}

adj_list = {
    1: [2, 3, 8, 4],
    2: [1, 3, 7, 5],
    3: [1, 2, 6],
    4: [3, 2, 1],
    5: [1],
    6: [1, 2, 3],
    7: [8, 1, 3],
    8: []
}

print("Out degrees:", get_out_degree(adj_list))
print("In degrees:", get_in_degree(adj_list))

# Output:
# Out degrees: {1: 4, 2: 4, 3: 3, 4: 3, 5: 1, 6: 3, 7: 3, 8: 0}
# In degrees: {1: 6, 2: 4, 3: 5, 4: 1, 5: 1, 6: 1, 7: 1, 8: 2}
    \end{lstlisting}

    \item \textbf{22.1-7 p. 593}.
    
    If we take the product of $B$ and $B^T$ for each entry $BB^{T}_{ij}$ we'd get:

    The sum of veticies value $B_i$ for each edge in the row with each edges vertex from $B^T_j$.
    Which if vertex $B_i = B^T_j$ would be 1 by 1 or -1 by -1 so we'd always get a positive number which is the total of the in degree and out degrees.

    However if we are comparing to differnt vertecies we'd be summing up the edges with each vertex. Therefore we'd have:

    \begin{equation}
      BB^T_{ij} =
      \begin{cases}
        \text{sum of in and out degrees,} & \text{if i = j,} \\
        \text{sum of edges between i and j,} & \text{otherwise,}
      \end{cases}
    \end{equation}

  \end{enumerate} % End Questions

\end{document}
